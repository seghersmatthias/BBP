%%=============================================================================
%% Inleiding
%%=============================================================================

\chapter{Inleiding}
\label{ch:inleiding}

De inleiding moet de lezer alle nodige informatie verschaffen om het onderwerp te begrijpen zonder nog externe werken te moeten raadplegen \autocite{Pollefliet2011}. Dit is een doorlopende tekst die gebaseerd is op al wat je over het onderwerp gelezen hebt (literatuuronderzoek).

Je verwijst bij elke bewering die je doet, vakterm die je introduceert, enz. naar je bronnen. In \LaTeX{} kan dat met het commando \texttt{$\backslash${textcite\{\}}} of \texttt{$\backslash${autocite\{\}}}. Als argument van het commando geef je de ``sleutel'' van een ``record'' in een bibliografische databank in het Bib\TeX{}-formaat (een tekstbestand). Als je expliciet naar de auteur verwijst in de zin, gebruik je \texttt{$\backslash${}textcite\{\}}.
Soms wil je de auteur niet expliciet vernoemen, dan gebruik je \texttt{$\backslash${}autocite\{\}}. Hieronder een voorbeeld van elk.

\textcite{Knuth1998} schreef een van de standaardwerken over sorteer- en zoekalgoritmen. Experten zijn het erover eens dat cloud computing een interessante opportuniteit vormen, zowel voor gebruikers als voor dienstverleners op vlak van informatietechnologie~\autocite{Creeger2009}.

\section{Stand van zaken}
\label{sec:stand-van-zaken}

%% TODO: deze sectie (die je kan opsplitsen in verschillende secties) bevat je
%% literatuurstudie. Vergeet niet telkens je bronnen te vermelden!


De onderzoeken en technieken om de noden van gebruikers te voorspellen kent de laatste jaren een grote opmars. Google en Facebook zijn dan ook volop bezig met eigen onderzoekscentra en technologieën te ontwikkelen. FAIR is de afkorting voor Facebook Artificial Intelligence Research er zijn 3 labo's in de wereld die constant opzoek zijn naar nieuwe mogelijkheden binnen AI. RankBrain is een algoritme van Google die a.h.v. artificiële intelligentie de ranking op de zoekpagina bepaald.
De persoonlijke advertenties die Google toont zijn veelal gegenereerd met algoritmes.  Maar ook gerelateerde producten of "wat jou ook kan interesseren\"-lijsten worden dikwijls door machine learning gegenereerd. Tensorflow \autocite{tensorflow} is een open source API ontwikkelt door Google die je uiteraard gratis kan gebruiken om zelf artificiële toepassingen te maken. De API is ontwikkeld in Python, deze taal is dan ook een van de veel gebruikte in de data science.

In deze bachelorproef zullen er voorspellingen gemaakt worden op een arcademachine. ToThePoint zal het eerste bedrijf zijn die AI in een arcademachine zal verwerken. Aangezien er nog geen andere gelijkaardige voorbeelden te vinden zijn zal ervan bij het begin onderzoek moeten geworden hoe we het best zouden aanpakken en verklaren. De spelletjes worden bestuurd door zes knoppen en een joystick. Aan de hand van de snelheid, aantal keer een knop of joystickbeweging is gedaan zullen we kunnen onderscheiden welk spel de gebruiker aan het spelen is. Er bestaan al veel verschillende soorten algoritmen maar deze zijn niet allemaal geschikt voor deze casus en daar zal dus aan gewerkt moeten worden. 

Artificiële Intelligentie is hot en zo komen er allerlei beginnende frameworks naar boven die reeds geïmplementeerde algoritmen bevatten of het eenvoudiger maken om ermee te starten. 
Tensorflow is een populair framework ontwikkelt door Google gemaakt in Python. Dit wordt gebruikt in verschillende producten van Google zoals, Google's stemherkenning, Google vertaler, het zoeken op afbeeldingen, etc.
TensorFlow werkt aan de hand van deep learning of neurale netwerken. De voorbeelden van Tensorflow gaan bijna uitsluitend over image recognition, het is mogelijk om andere applicaties ermee te maken. Dit is dan ook de reden dat Google dit framework open source heeft gemaakt zodat ze kunnen zien waar er nog verbeteringen kunnen aangebracht worden en wat er nog allemaal mogelijk is.
Een ander interessant framework is het accord-framework \autocite{accord} die volledig ontwikkeld is in C\#. Hierin zitten veel geïmplementeerde algoritmen in die dan kan gebruikt worden door derden om een eigen applicatie te ontwikkelen. 



\section{Probleemstelling en Onderzoeksvragen}
\label{sec:onderzoeksvragen}

%% TODO:
%% Uit je probleemstelling moet duidelijk zijn dat je onderzoek een meerwaarde
%% heeft voor een concrete doelgroep (bv. een bedrijf).
%%
%% Wees zo concreet mogelijk bij het formuleren van je
%% onderzoeksvra(a)g(en). Een onderzoeksvraag is trouwens iets waar nog
%% niemand op dit moment een antwoord heeft (voor zover je kan nagaan).

Machine learning is momenteel aan het boomen. Dit wordt nu zeer veel toegepast voor online advertising met Google en Facebook als de leiders. Maar ook meer en meer bedrijven beginnen zich te verdiepen in artificiële intelligentie. ToThePoint is zich hierop ook aan het voorbereiden. Doormiddel van een funproject willen ze zoveel als mogelijk bijleren op allerlei gebieden binnen de informatica. 

Men heeft een arcade machine gekocht die ze volledig gaan customizen met verschillende technologieën Op deze manier kunnen ze al hun kennis loslaten. Inclusief de kennis die ze zullen opdoen via deze bachelorproef. Hierdoor zullen ze dus iets bijleren over artificiële intelligentie en dit dan ook kunnen toepassen. Verder zal de machine geplaatst worden op allerlei jobbeurzen om gegevens van studenten op te slaan bijvoorbeeld. Het nut van de machine is niet alleen om bij te leren maar hij zal ook dienen als referentie naar klanten toe. Op deze manier kan ToThePoint aantonen tot wat ze instaat zijn. 

In deze bachelorproef wordt er een vergelijkende studie gemaakt over twee verschillende algoritmen. Het beste algoritme zal dan uiteindelijk geïmplementeerd worden in de arcade machine. Je kan hier heel ver in gaan. De voorspellingen kunnen bijvoorbeeld gemaakt worden door wat de meest gebruikte knop is of joystick beweging. Maar door de snelheid dat knoppen bediend worden en de frequentie van dezelfde knop kan bepaald worden welk spel er gespeeld wordt. Als het nog verder uitgewerkt wordt kan er naar toetsencombinaties gekeken worden maar dit is te vergaand voor deze bachelorproef. Er zal vooral gefocust worden op hoe een goed algoritme gekozen wordt. Het stappenplan die uitgewerkt zal worden kan dan ook toegepast worden op toekomstige projecten.



\section{Opzet van deze bachelorproef}
\label{sec:opzet-bachelorproef}

%% TODO: Het is gebruikelijk aan het einde van de inleiding een overzicht te
%% geven van de opbouw van de rest van de tekst. Deze sectie bevat al een aanzet
%% die je kan aanvullen/aanpassen in functie van je eigen tekst.

De rest van deze bachelorproef is als volgt opgebouwd:

In Hoofdstuk~\ref{ch:methodologie} wordt de methodologie toegelicht en worden de gebruikte onderzoekstechnieken besproken om een antwoord te kunnen formuleren op de onderzoeksvragen.

%% TODO: Vul hier aan voor je eigen hoofstukken, één of twee zinnen per hoofdstuk

In Hoofdstuk~\ref{ch:conclusie}, tenslotte, wordt de conclusie gegeven en een antwoord geformuleerd op de onderzoeksvragen. Daarbij wordt ook een aanzet gegeven voor toekomstig onderzoek binnen dit domein.

