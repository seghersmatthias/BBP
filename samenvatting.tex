%%=============================================================================
%% Samenvatting
%%=============================================================================

%% TODO: De "abstract" of samenvatting is een kernachtige (~ 1 blz. voor een
%% thesis) synthese van het document.
%%
%% Deze aspecten moeten zeker aan bod komen:
%% - Context: waarom is dit werk belangrijk?
%% - Nood: waarom moest dit onderzocht worden?
%% - Taak: wat heb je precies gedaan?
%% - Object: wat staat in dit document geschreven?
%% - Resultaat: wat was het resultaat?
%% - Conclusie: wat is/zijn de belangrijkste conclusie(s)?
%% - Perspectief: blijven er nog vragen open die in de toekomst nog kunnen
%%    onderzocht worden? Wat is een mogelijk vervolg voor jouw onderzoek?
%%
%% LET OP! Een samenvatting is GEEN voorwoord!

%%---------- Nederlandse samenvatting -----------------------------------------
%%
%% TODO: Als je je bachelorproef in het Engels schrijft, moet je eerst een
%% Nederlandse samenvatting invoegen. Haal daarvoor onderstaande code uit
%% commentaar.
%% Wie zijn bachelorproef in het Nederlands schrijft, kan dit negeren en heel
%% deze sectie verwijderen.

\IfLanguageName{english}{%
\selectlanguage{dutch}
\chapter*{Samenvatting}
\lipsum[1-4]
\selectlanguage{english}
}{}

%%---------- Samenvatting -----------------------------------------------------
%%
%% De samenvatting in de hoofdtaal van het document

\chapter*{\IfLanguageName{dutch}{Samenvatting}{Abstract}}

%\lipsum[1-4]
Artificiële intelligentie is zich steeds meer en meer aan het ontwikkelen. Grote bedrijven als Google, Facebook, ... zijn daar al volop mee bezig. Maar ook kleinere bedrijven beginnen te experimenteren met machine learning. ToThePoint is zo'n bedrijf. Zij hebben een arcademachine die ze helemaal ombouwen met allerlei nieuwe technologieën in de informatica. Artificiële intelligentie maakt daar deel vanuit. Aan de hand van deze arcademachine wilt ToThePoint tonen aan potentiële klanten tot wat ze allemaal in staat zijn. Wanneer mensen op de arcademachine spelen is het de bedoeling dat er op een schermpje naast de machine een boodschap komt met een voorspelling welk spel de speler speelt. Op basis van verschillende parameters zal het mogelijk zijn die voorspelling te doen. De gebruiker wilt natuurlijk niet te lang wachten op die voorspelling of men wil ook niet Tetris zien verschijnen terwijl ze eigenlijk Mortal Kombat spelen. Daarom is het belangrijk dat een snel maar accuraat algoritme gekozen wordt. Tijdens dit onderzoek gaan we opzoek naar het beste algoritme. Na een vergelijkende studie tussen logistische regressie en support vector machine zal het optimale algoritme gebruikt kunnen worden in de implementatie van de arcademachine. Er bestaan veel verschillende machine learning algoritmen. Een eerste belangrijke stap was dus om twee goede algoritmen te vinden en die uitgebreid te bespreken hoe die in elkaar zitten. Daarna wordt logistische regressie en support vector machine geïmplementeerd met het Accord-framework. De implementatie zal stap voor stap gebeuren. Beginnend met voorspellingen tussen twee spelletjes met twee inputparameters. Om tenslotte te eindigen met vier spelletjes en vijf inputparameters. Het snelste algoritme is logistische regressie. Support vector machine is dan een stuk beter in de juiste voorspellingen. De fouten die een SVM(Support vector machine) maakt zijn ook logisch te verklaren in tegenstelling tot logistische regressie. Logistische regressie maakt bijvoorbeeld fouten tussen Mortal Kombat en Tetris wat eigenlijk totaal verschillende spelletjes zijn. SVM maakt enkel fouten tussen Pacman en Arkanoid wat wel te verstaan is want die spelletjes maken enkel en alleen gebruik van een joystick. Support vector machine heeft gemiddeld 169,4067 ms nodig om een hypothese te vormen, $\pm$ 6 keer zoveel als logistische regressie. Een mens zal dit verschil nooit merken maar de effectieve voorspellingen zal een gebruiker wel onthouden. Daarom krijgt support vector machine de voorkeur om in de arcademachine geïmplementeerd te worden
